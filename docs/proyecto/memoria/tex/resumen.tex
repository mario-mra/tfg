\begin{resumen}

El objetivo del presente Trabajo Fin de Grado, consiste en el desarrollo e implementación de un sistema \emph{hardware} capaz de obtener, y posteriormente almacenar, tramas de bus \emph{USB} (del inglés, \emph{Universal Serial Bus}, o en español, Bus Serie Universal). Se busca a su vez, que dicho sistema sea lo más económico posible, utilizando únicamente herramientas y utilidades de código libre.

Previamente a su elaboración, se ha realizado un estudio del estado del arte en cuanto a sistemas de captación y análisis tanto \emph{hardware} como \emph{software}, observando el elevado precio de estos, reafirmando la necesidad de disponer un sistema de bajo coste.

Para su desarrollo se ha utilizado en primer lugar una placa de evaluación que incluye el circuito integrado especializado \emph{USB3300}, encargado de la propia capa física del bus USB. Esta placa también incluye dos conectores hembra USB, uno de tipo A y el otro de tipo mini-B, a los cuales se conectan ambos extremo del bus a analizar.

Seguidamente, se ha usado la placa de desarrollo \emph{iCEstick}, que entre otros componentes, incluye una \emph{FPGA iCE40HX1k} de la compañía \emph{Lattice} y un conversor de USB a puerto serie \emph{FTDI FT2232HL}. Dicha placa, tiene la función de comunicarse tanto con el integrado anterior encargado de la capa física, como con el equipo de control, siendo este un PC.

A partir de este punto, el desarrollo del sistema se ramifica en dos grupos diferenciados, el primero encargado de todo lo referente a la FPGA, y el segundo encargado del software de control. Para su realización se sigue un procedimiento ágil basado en \emph{Scrum}, que lo divide en iteraciones de varias semanas denominadas \emph{sprints}.

Para el primer grupo, se define previamente una serie de metodologías con las que poder desarrollar de forma efectiva, limpia y segura los diversos módulos necesarios. De esta forma, se consigue programar en lenguaje descriptor de \emph{hardware} Verilog varios módulos con funcionalidades muy diversas, entre los que se encuentra:
\begin{itemize}
    \item \textbf{Módulo de comunicación serie.} Cuya función es la de procesar y generar señales, de entrada y salida respectivamente, que permitan una comunicación serie con el PC.
    \item \textbf{Módulo de comunicación ULPI.} Cuya función, igual que en el caso anterior, es la de procesar y generar señales, pero en este caso para el bus ULPI generado por el integrado \emph{USB3300}, tal como dicta su especificación. Gracias a este módulo, el sistema es capaz por un lado de escribir y leer registros, y por otro, de obtener los datos USB.
    \item \textbf{Módulo de almacenaje FIFO.} Cuya función es la de generar una memoria \emph{FIFO} (\emph{First-In First-Out}) a partir de los bloques de RAM internos disponibles en la FPGA. La peculiaridad de esta memoria es la forma en la que almacena y vuelca los datos, siendo el primer dato almacenado el primero en ser devuelto.
    \item \textbf{Módulo de control general.} Cuya función es la unir y gestionar todos los módulos implicados, permitiendo un funcionamiento sincronizado entre ellos.
\end{itemize}
Todos ellos, han soportado varias simulaciones con las que asegurar su correcto funcionamiento antes de su despliegue en el sistema final. \\
Junto con los anteriores módulo, se desarrolla un protocolo simple con el que poder comandar el sistema. Este protocolo esta formado por dos bytes, en los que se incluyen el comando a realizar, una dirección de 6 bits, y un byte de datos. \\
Hay que comentar también el diseño e impresión de una pequeña base en 3D, donde situar todos los componentes de forma segura.

El segundo grupo, aun teniendo una menor dificultad respecto al primero, es esencial para poder ver los resultados obtenidos. Para controlar la totalidad del sistema se ha creado una aplicación en lenguaje C, que por medio de un simple menú permita la interacción entre la FPGA y el usuario. Este menú permite por un lado configurar y abrir el puerto al que está conectado la FPGA, y por otro, permite enviar los comandos necesarios para hacer funcionar al sistema (siendo estos de escritura de registros, lectura de registros y activación/desactivación del envío de datos al PC) y recibir los datos USB capturados. \\
Esta aplicación también se encarga de almacenar de forma estructurada dichos datos en un archivo formato \emph{JSON}, que posteriormente puede ser estudiado. Secundariamente, y tras la finalización de todo lo anterior, se ha creado una pequeña utilidad, que convierte el fichero \emph{JSON} a \emph{PCAP}, lo que permite ser visualizado en el programa \emph{Wireshark}.

Hay que destacar que todo el sistema en su conjunto ha costado menos de 45\$ (aproximadamente 40\texteuro), precio muy por debajo de cualquier otro método de captura comercial.

Para finalizar, hay que comentar varios aspectos a incluir o mejorar.
\begin{itemize}
    \item Mejorar el método de transmisión hacia el PC, para aumentar su velocidad de transferencia, y evitar posibles perdidas de datos. Por ejemplo, se puede prescindir del puerto serie, y utilizar otro protocolo soportado por el integrado FTDI, que permita una mayor tasa de transferencia.
    \item Incorporar memoria RAM externa, que aumente significativamente la cantidad de datos que puede almacenar temporalmente el sistema.
    \item Crear un disector para \emph{Wireshark}, que sea capaz de dar más información sobre los datos capturados.
\end{itemize}

\end{resumen}


\begin{abstract}
The main purpose of this Final Degree Project is the development and implementation of a hardware base system capable of capture and store a USB (Universal Serial Bus) data frame. Furthermore, this system has to be as affordable as possible, using only open source tools and utilities.

Previously to the elaboration of the project, a study on the current State of the Art of hardware and software based capture and analysis systems has been made. Noting the high price of them, the necessity of having an inexpensive system were reaffirmed.

To obtain the desired result, a evaluation board containing the USB3300 specialised integrated circuit has been used, that integrated circuit handles the Physical layer of the bus, reducing the complexity level of the final design. This board also includes two USB female connectors, one of type A and the other one of type mini-B, where both ends of the USB bus to analyze are connected. It is worth mentioning that this integrated circuit uses the parallel protocol ULPI (UTMI+ Low Pin Interface) to interface with other devices.

The iCEstick development board, that includes the FPGA (Field-Programmable Gate Array) iCE40HX1K from the Lattice Semiconductor manufacturer and a the FTDI FT2232HL USB to Serial converter integrated circuit, is then use to communicate with both the USB3300 and the control equipment, in this case a PC.


\noWord[Terminar!!]
\end{abstract}
