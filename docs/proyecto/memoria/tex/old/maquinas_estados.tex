\begin{figure}
    \centering
    \begin{tikzpicture}
        \tikzFSM
        \node[state, initial, accepting] (IDLE) {SERIAL\_IDLE};
        \node[state, below of=IDLE] (READ) {SERIAL\_READ};
        \node[state, below of=READ] (PUSH) {SERIAL\_PUSH};
        
        \draw (IDLE) edge[trs, bend right=15] node[left]{!UART\_Rx\_EMPTY} (READ)
              (IDLE) edge[trs, loop above] node[above]{UART\_Rx\_EMPTY} (IDLE);
        
        \draw (READ) edge[trs, below] node[right]{n\_byte\_r = 1} (PUSH)
              (READ) edge[trs, bend right=15] node[right]{n\_byte\_r != 1} (IDLE);

        \draw (PUSH) edge[trs, bend right=80] node[right]{1} (IDLE);

    \end{tikzpicture}
    \caption{Máquina de estados del módulo ULPI\_op\_stack}
\end{figure}
% https://tex.stackexchange.com/questions/305375/how-to-position-labels-at-edges-of-tikz-figures
% sloped
% https://tex.stackexchange.com/questions/35423/changing-loop-size-and-shape-in-tikz#35426
\begin{figure}
    \centering
    % \scalebox{0.75}
    % {
    \begin{tikzpicture}
        \tikzFSM;
        \node[state, initial, accepting] (START) {ULPI\_START};
        \node[state, below of=START] (IDLE) {ULPI\_IDLE};
        \node[state] at ($ (IDLE)+(-90:4) $)(RECV) {ULPI\_RECV};
        \node[state] at ($ (IDLE)+(-150:4) $) (REGW) {ULPI\_REG\_WRITE};
        \node[state] at ($ (IDLE)+(-30:4) $) (REGR) {ULPI\_REG\_READ};

        \draw[trs]
            (START) edge[below] node[right]{!DIR} (IDLE)
            (START) edge[loop above] node[above]{DIR} (START)
        
            (IDLE) edge[bend right=15] node[left]{RECV\_busy} (RECV)
            (IDLE) edge[bend right=15] node[above, sloped]{ctrl\_PrW} (REGW)
            (IDLE) edge[bend right=15] node[above, sloped]{ctrl\_PrR} (REGR)
            (IDLE) edge[in=150, out=120, loop] node[above left]{DIR} (IDLE)

            (RECV) edge[bend right=15] node[right]{!RECV\_busy} (IDLE)
            (RECV) edge[loop below] node[below]{RECV\_busy} (RECV)

            (REGW) edge[bend right=15] node[above, sloped]{!RW\_busy} (IDLE)
            (REGW) edge[loop below] node[below]{RW\_busy} (REGW)

            (REGR) edge[bend right=15] node[above, sloped]{!RR\_busy} (IDLE)
            (REGR) edge[loop below] node[below]{RR\_busy} (REGR);
    \end{tikzpicture}
    % }
    \caption{Máquina de estados del módulo ULPI}
\end{figure}


\begin{figure}
    \centering
    \begin{tikzpicture}
        \tikzFSM;
        \node[state, initial, accepting] (IDLE) {ULPI\_RECV\_IDLE};
        \node[state] at ($(IDLE)+(0,-4.5)$) (READ) {ULPI\_RECV\_READ};

        \node[infoNode] at ($(IDLE)+(4.5,-1)$) (infoA) {\begin{tabular}{c}
                                                    \textbf{A\textsuperscript{*}} \\
                                                    \hline \\
                                                    (ctrl\_DATA | ctrl\_RxCMD) \& \\
                                                    ReadAllow
                                                \end{tabular}};
        \node[infoNode] at ($(infoA)+(0,-2.5)$) (infoB) {\begin{tabular}{c}
                                                            \textbf{B\textsuperscript{*}} \\
                                                            \hline \\
                                                            ctrl\_DATA | \\
                                                            ctrl\_RxCMD
                                                        \end{tabular}};

        \draw[trs]
            (IDLE) edge[bend right=15] node[left]{A\textsuperscript{*}} (READ)
            (IDLE) edge[loop above] node[above]{!A\textsuperscript{*}} (IDLE)

            (READ) edge[bend right=15] node[right]{!B\textsuperscript{*}} (IDLE)
            (READ) edge[loop below] node[below]{B\textsuperscript{*}} (READ);
    \end{tikzpicture}
    \caption{Máquina de estados del módulo ULPI\_RECV}
\end{figure}


\begin{figure}
    \centering
    \begin{tikzpicture}
        \tikzFSM;
        \node[state, initial, accepting] (IDLE) {ULPI\_RR\_IDLE};
        \node[state, below of=IDLE] (TXCMD) {ULPI\_RR\_TXCMD};
        \node[state, below of=TXCMD] (WAIT1) {ULPI\_RR\_WAIT1};
        \node[state, below of=WAIT1] (READ) {ULPI\_RR\_READ};
        \node[state, below of=READ] (WAIT2) {ULPI\_RR\_WAIT2};

        \draw[trs]
            (IDLE) edge[] node[left]{PrR} (TXCMD)
            (IDLE) edge[loop above] node[above]{!PrR} (IDLE)

            (TXCMD) edge[] node[left]{NXT\_r} (WAIT1)
            (TXCMD) edge[loop left] node[left]{!NXT\_r} (TXCMD)

            (WAIT1) edge[] node[left]{1} (READ)

            (READ) edge[] node[left]{1} (WAIT2)

            (WAIT2) edge[bend right] node[right]{1} (IDLE);
    \end{tikzpicture}
    \caption{Máquina de estados del módulo ULPI\_REG\_READ}
\end{figure}


\begin{figure}
    \centering
    \begin{tikzpicture}
        \tikzFSM;
        \node[state, initial, accepting] (IDLE) {ULPI\_RW\_IDLE};
        \node[state, below of=IDLE] (TXCMD) {ULPI\_RW\_TXCMD};
        \node[state, below of=TXCMD] (DATA) {ULPI\_RW\_DATA};
        \node[state, below of=DATA] (STOP) {ULPI\_RW\_STOP};

        \draw[trs]
            (IDLE) edge[] node[left]{PrW} (TXCMD)
            (IDLE) edge[loop above] node[above]{!PrW} (IDLE)

            (TXCMD) edge[] node[left]{NXT\_r} (DATA)
            (TXCMD) edge[loop left] node[left]{!NXT\_r} (TXCMD)

            (DATA) edge[] node[left]{NXT\_r} (STOP)
            (DATA) edge[loop left] node[left]{!NXT\_r} (DATA)

            (STOP) edge[bend right=35] node[right]{1} (IDLE);
    \end{tikzpicture}
    \caption{Máquina de estados del módulo ULPI\_REG\_WRITE}
\end{figure}


\begin{figure}
    \centering
    \begin{tikzpicture}
        \tikzFSM;
        \node[state, initial, accepting] (IDLE) {Rx\_IDLE};
        \node[state, below of=IDLE] (INIT) {Rx\_INIT};
        \node[state, below of=INIT] (RECV) {Rx\_RECV};
        \node[state, below of=RECV] (SAVE) {Rx\_SAVE};

        \draw[trs, node distance=2.2cm]
            (IDLE) edge[] node[left]{!Rx} (INIT)
            (IDLE) edge[loop above] node[above]{Rx} (IDLE)

            (INIT) edge[] node[left]{1} (RECV)

            (RECV) edge[] node[left]{Rx\_counter\_r = 10} (SAVE)
            (RECV) edge[loop left] node[left]{Rx\_counter\_r != 10} (RECV)

            (SAVE) edge[bend right] node[right]{1} (IDLE);
    \end{tikzpicture}
    \caption{Máquina de estados del módulo UART\_Rx}
\end{figure}


\begin{figure}
    \centering
    \begin{tikzpicture}
        \tikzFSM;
        \node[state, initial, accepting] (IDLE) {Tx\_IDLE};
        \node[state, below of=IDLE] (LOAD) {Tx\_LOAD};
        \node[state, below of=LOAD] (TRANS) {Tx\_TRANS};

        \draw[trs, node distance=2.2cm]
            (IDLE) edge[] node[left]{!rd\_empty} (LOAD)
            (IDLE) edge[loop above] node[above]{rd\_empty} (IDLE)

            (LOAD) edge[] node[left]{1} (TRANS)

            (TRANS) edge[bend right] node[right]{Tx\_counter\_r = 11} (IDLE)
            (TRANS) edge[loop left] node[left]{Tx\_counter\_r != 11} (TRANS);
    \end{tikzpicture}
    \caption{Máquina de estados del módulo UART\_Tx}
\end{figure}


\begin{figure}
    \centering
    \scalebox{0.9}
    {
    \begin{tikzpicture}
        \tikzFSM;
        \newcommand{\fsmBaseAng}{0};
        \newcommand{\fsmInfoSep}{2.5};
        \newcommand{\nuSpc}{\ensuremath{\,}}

        \node[state, accepting, minimum width={width("MAIN\_IDLE")+.75cm}] (IDLE) {MAIN\_IDLE};

        \node[state] at ($(IDLE)+(\fsmBaseAng+0:5)$) (RECVT) {MAIN\_RECV\_TOGGLE};

        \node[state] at ($(IDLE)+(\fsmBaseAng+60:5)$) (RSEND) {MAIN\_REG\_SEND};
        \node[state] at ($(IDLE)+(\fsmBaseAng+120:5)$) (FSEND) {MAIN\_FORCE\_SEND};
        \node[state] at ($(IDLE)+(\fsmBaseAng+90:8)$) (UWAIT) {MAIN\_UART\_WAIT};
        
        \node[state] at ($(IDLE)+(\fsmBaseAng+180:5)$) (RREAD) {MAIN\_REG\_READ};
        \node[state] at ($(IDLE)+(\fsmBaseAng+240:5)$) (RWRITE) {MAIN\_REG\_WRITE};
        \node[state] at ($(IDLE)+(\fsmBaseAng+210:7)$) (RWAIT) {MAIN\_REG\_WAIT};
        
        \node[state] at ($(IDLE)+(\fsmBaseAng+300:5)$) (RECV) {MAIN\_RECV};
        \node[state] at ($(RECV)+(0,-3.75)$) (RECVS1) {MAIN\_RECV\_SEND1};
        \node[state] at ($(RECVS1)+(4.25,0)$) (RECVWAIT) {MAIN\_RECV\_WAIT};
        \node[state] at ($(RECVWAIT)+(0,+3.75)$) (RECVS2) {MAIN\_RECV\_SEND2};

        % Alternative initial arrow with custom entry angle
        \node[] at ($(IDLE)+(\fsmBaseAng+150:2.5)$) (altStart) {};
        \draw[startSty] (altStart) edge[] node[]{} (IDLE);


        \matrix [row sep={0.3cm,between borders},
                 column sep={0.5cm,between borders},
                 ampersand replacement=\?] 
                 at ($ (IDLE)+(0, -12.5) $) (InfoIdle) {
                    \node[infoNode] (infoI0) {\begin{tabular}{c}
                                                 \textbf{OP\textsubscript{toggle}\textsuperscript{*}} \\
                                                 \hline \\
                                                 !op\_stack\_empty \& \\
                                                 op\_cmd = 00
                                              \end{tabular}}; \?
                    \node[infoNode] (infoI1) {\begin{tabular}{c}
                                                 \textbf{OP\textsubscript{RegSend}\textsuperscript{*}} \\
                                                 \hline \\
                                                 !op\_stack\_empty \& \\
                                                 op\_cmd = 01
                                              \end{tabular}}; \?
                    \node[infoNode] (infoI2) {\begin{tabular}{c}
                                                 \textbf{OP\textsubscript{RegWrite}\textsuperscript{*}} \\
                                                 \hline \\
                                                 !op\_stack\_empty \& \\
                                                 op\_cmd = 10
                                              \end{tabular}}; \?
                    \node[infoNode] (infoI3) {\begin{tabular}{c}
                                                 \textbf{OP\textsubscript{RegRead}\textsuperscript{*}} \\
                                                 \hline \\
                                                 !op\_stack\_empty \& \\
                                                 op\_cmd = 11
                                              \end{tabular}}; \\
                                              \?
                    \node[infoNode] (infoI4) {\begin{tabular}{c}
                                                 \textbf{OP\textsubscript{recv}\textsuperscript{*}} \\
                                                 \hline \\
                                                 op\_stack\_empty \& \\
                                                 !ULPI\_INFO\_buff\_empty \& \\
                                                 toggle\_r
                                              \end{tabular}}; \?
                    \node[infoNode] (infoI5) {\begin{tabular}{c}
                                                 \textbf{OP\textsubscript{test}\textsuperscript{*}} \\
                                                 \hline \\
                                                 op\_stack\_empty \& \\
                                                 (ULPI\_INFO\_buff\_empty | \\
                                                 !toggle\_r) \& \\
                                                 force\_send
                                              \end{tabular}}; \\
                  };
        \matrix [row sep={0.3cm,between borders},
                 column sep={0.5cm,between borders},
                 ampersand replacement=\?] 
                 at ($ (IDLE)+(7, 5) $) (InfoOther) {
                    \node[infoNode] (info0) {\begin{tabular}{c}
                                                \textbf{OP\textsubscript{RecvSend1}\textsuperscript{*}} \\
                                                \hline \\
                                                UART\_Tx\_FULL | \\
                                                INFO\_count\_r != 0
                                             \end{tabular}}; \\
                    \node[infoNode] (info1) {\begin{tabular}{c}
                                                \textbf{OP\textsubscript{RecvReturn}\textsuperscript{*}} \\
                                                \hline \\
                                                !UART\_Tx\_FULL \& \\
                                                DATA\_count\_r = 0
                                             \end{tabular}}; \\
                    \node[infoNode] (info2) {\begin{tabular}{c}
                                                \textbf{OP\textsubscript{RecvSend2}\textsuperscript{*}} \\
                                                \hline \\
                                                UART\_Tx\_FULL | \\
                                                DATA\_count\_r != 0
                                             \end{tabular}}; \\
                  };


        \draw[trs]
            (IDLE) edge[bend right=20] node[sloped, above]{OP\textsubscript{toggle}\textsuperscript{*}} (RECVT)
            (IDLE) edge[] node[sloped, above]{OP\textsubscript{RegSend}\textsuperscript{*}} (RSEND)
            (IDLE) edge[] node[sloped, above]{OP\textsubscript{test\textsuperscript{*}}} (FSEND)
            (IDLE) edge[] node[sloped, above]{OP\textsubscript{RegRead}\textsuperscript{*}} (RREAD)
            (IDLE) edge[] node[sloped, above]{OP\textsubscript{RegWrite}\textsuperscript{*}} (RWRITE)
            (IDLE) edge[] node[sloped, above]{OP\textsubscript{recv}\textsuperscript{*}} (RECV)
            (IDLE) edge[loop below] node[below, text width=1.5cm]{!OP\textsubscript{toggle}\textsuperscript{*}\nuSpc \&
                                                                  !OP\textsubscript{RegSend}\textsuperscript{*}\nuSpc \&
                                                                  !OP\textsubscript{test}\textsuperscript{*}\nuSpc \& 
                                                                  !OP\textsubscript{RegRead}\textsuperscript{*}\nuSpc \& 
                                                                  !OP\textsubscript{RegWrite}\textsuperscript{*}\nuSpc \& 
                                                                  !OP\textsubscript{recv}\textsuperscript{*}\nuSpc} (IDLE)
            
            (RSEND) edge[] node[right]{1} (UWAIT)

            (FSEND) edge[] node[left]{1} (UWAIT)

            (UWAIT) edge[] node[sloped, above]{!UART\_Tx\_FULL} (IDLE)
            (UWAIT) edge[loop above] node[above]{UART\_Tx\_FULL} (UWAIT)
            
            (RREAD) edge[] node[left]{1} (RWAIT)
            
            (RWRITE) edge[] node[below]{1} (RWAIT)

            (RWAIT) edge[] node[sloped, above]{!ULPI\_busy} (IDLE)
            (RWAIT) edge[out={\fsmBaseAng+210+15}, in={\fsmBaseAng+210-15}, loop] node[below=0.15]{ULPI\_busy} (RWAIT)

            (RECV) edge[] node[left]{1} (RECVS1)

            (RECVS1) edge[] node[below]{!OP\textsubscript{RecvSend1}\textsuperscript{*}} (RECVWAIT)
            (RECVS1) edge[loop left] node[]{OP\textsubscript{RecvSend1}\textsuperscript{*}} (RECVS1)

            (RECVWAIT) edge[] node[right, text width=2.5cm]{!UART\_Tx\_FULL \& DATA\_count\_r != 0} (RECVS2)
            (RECVWAIT) edge[bend right=10] node[sloped, above]{OP\textsubscript{RecvReturn}\textsuperscript{*}} (IDLE)
            (RECVWAIT) edge[loop right] node[text width=2.5cm]{UART\_Tx\_FULL} (RECVWAIT)

            (RECVS2) edge[bend right=10] node[sloped, above]{!OP\textsubscript{RecvSend2}\textsuperscript{*}} (IDLE)
            (RECVS2) edge[loop right] node[]{OP\textsubscript{RecvSend2}\textsuperscript{*}} (RECVS2)

            (RECVT) edge[bend right=20] node[above]{1} (IDLE);
    \end{tikzpicture}
    }
    \caption{Máquina de estados del módulo main\_controller}
\end{figure}