\chapter{Manual de instalación y utilización}
\label{ch:manual}
%* Comprobar
En el presente anexo, se detallan los procedimientos a seguir para la instalación y utilización del sistema bajo un sistema Linux.

%* Comprobar
\section{Requisitos \emph{software} previos}
Antes de poder generar un archivo binario compatible con la FPGA \emph{Lattice iCE40HX1K}, es necesario instalar las aplicaciones encargadas de su creación.

\subsubsection{Dependencidas de las aplicaciones}
Previo a la instalación de dichas aplicaciones, hay que instalar también sus dependencias. A continuación se dan la lista de paquetes a instalar según tres sistemas operativos Linux distintos.

\begin{itemize}
    \item \textbf{Para una instalación en \emph{Ubuntu}} \\
    \begin{verbatim}
$ sudo apt-get install build-essential clang bison flex libreadline-dev \
                       gawk tcl-dev libffi-dev git mercurial graphviz   \
                       xdot pkg-config python python3 libftdi-dev       \
                       qt5-default python3-dev libboost-all-dev cmake}
    \end{verbatim}

    \item \textbf{Para una instalación en \emph{Fedora}} \\
    \begin{verbatim}
$ sudo dnf install make automake gcc gcc-c++ kernel-devel clang bison            \
                   flex readline-devel gawk tcl-devel libffi-devel git mercurial \
                   graphviz python-xdot pkgconfig python python3 libftdi-devel   \
                   qt5-devel python3-devel boost-devel boost-python3-devel
    \end{verbatim}

    \item \textbf{Para una instalación en \emph{ArchLinux}} \\
    \begin{verbatim}
$ sudo pacman -S base-devel clang boost-libs python qt5-base boost \
                 cmake eigen git trellis libffi tcl xdot mercurial \
                 graphviz libftdi-compat 
    \end{verbatim}
\end{itemize}

\subsubsection{Instalación de las herramientas \emph{IceStorm}}
Se procede a instalar las herramientas y bases de datos del proyecto \emph{IceStorm} (\emph{icepack}, \emph{icebox}, \emph{iceprog}, \emph{icetime} y bases de datos de las diversas FPGAs).

\begin{lstlisting}[language=bash]
$ git clone https://github.com/cliffordwolf/icestorm.git icestorm
$ cd icestorm
$ make -j$(nproc)
$ sudo make install
\end{lstlisting}

\subsubsection{Instalación de \emph{NextPNR}}
Se procede a instalar la herramienta de posicionado y encaminamiento (\emph{Place-And-Route}) \emph{NextPNR}.

\begin{lstlisting}[language=bash]
$ git clone https://github.com/YosysHQ/nextpnr nextpnr
$ cd nextpnr
$ cmake -DARCH=ice40 -DCMAKE_INSTALL_PREFIX=/usr/local .
$ make -j$(nproc)
$ sudo make install
\end{lstlisting}

\subsubsection{Instalación de \emph{Yosys}}
Se procede a instalar la herramienta de síntesis del lenguaje Verilog \emph{Yosys}.

\begin{lstlisting}[language=bash]
$ git clone https://github.com/cliffordwolf/yosys.git yosys
$ cd yosys
$ make -j$(nproc)
$ sudo make install
\end{lstlisting}
% \begin{verbatim}
% $ git clone https://github.com/cliffordwolf/yosys.git yosys
% $ cd yosys
% $ make -j$(nproc)
% $ sudo make install
% \end{verbatim}

\subsubsection{Permisos de subida}
Si se desea subir el archivo binario generado sin utilizar permisos administrativos del sistema, se ha de ejecutar el siguiente comando.
\begin{lstlisting}[language=bash]
$ sudo sh -c 'echo "ATTRS{idVendor}=="0403", ATTRS{idProduct}=="6010", MODE="0660", GROUP="plugdev", TAG+="uaccess"" > /etc/udev/rules.d/53-lattice-ftdi.rules'
\end{lstlisting}


%* Comprobar
\section{Descarga del repositorio}
Para poder utilizar el sistema, hay que descargar en primer lugar los datos del repositorio.
\begin{lstlisting}[language=bash]
$ git clone https://github.com/mario-mra/tfg.git analizador_usb
$ cd analizador_usb
\end{lstlisting}
A partir de este momento, todas las operaciones a realizar estarán referencias al directorio del repositorio: \emph{.\textbackslash analizador\_usb}.


%* Comprobar
\section{Generación y programación del archivo binario}
Los archivos a utilizar para la generación del binario están situados en la carpeta \emph{./ICEstick/USB3300\_sniffer}. Una vez situados en esa carpeta, y si todos los requisitos están instalados, ya sería posible realizar la generación y posterior programación del archivo binario.

\begin{lstlisting}[language=bash]
$ cd ./ICEstick/USB3300_sniffer
$ make pnr
$ make prog
$ cd ../../
\end{lstlisting}


%* Comprobar
\section{Instalación de las aplicaciones de control}
Las aplicaciones creadas están situadas en la carpeta \emph{./PC/}. Previamente a su ejecución hay que compilarlas e instalarlas.

\subsubsection{Instalación de la aplicación de control}
\begin{lstlisting}[language=bash]
$ cd ./PC/app
$ make
$ sudo make install
$ cd ../../
\end{lstlisting}

\subsubsection{Instalación de la aplicación de conversión \emph{JSON} a \emph{PCAP}}
\begin{lstlisting}[language=bash]
$ cd ./PC/json2pcap
$ make
$ sudo make install
$ cd ../../
\end{lstlisting}


%* Comprobar
\section{Obtención de la captura}
Con la aplicación de control ya instalada, y el sistema conectado al PC, ya se podría comenzar la captura.
Para evitar posibles perdidas de datos capturados, se recomienda conectar el bus USB a analizar con el sistema en funcionamiento y la señal \emph{RECV\_TOGGLE} activa.

\emph{\textbf{Nota}}. La aplicación se encarga de escribir los registros ULPI necesarios para poder realizar la captura, por lo que las opciones \textbf{2 y 3} no son necesarias utilizarlas.

\begin{enumerate}
    \item Ejecutar la aplicación. \emph{\$ \textbf{FPGA-usb-capture}}.
    \item Configurar el puerto si fuera necesario, eligiendo el dispositivo a usar. \textbf{[Opción 0]}
    \item Abrir el puerto serie con la configuración dada. \textbf{[Opción 1]}
    \item Empezar la recepción de datos, activando la señal \emph{RECV\_TOGGLE}. \textbf{[Opción 4]}
    \item Conectar el dispositivo a analizar a la placa \emph{USB3300}.
    \item Realizar todas las operaciones deseadas.
    \item Cerrar el programa (automáticamente se apaga la señal \emph{RECV\_TOGGLE} y se cierra el puerto). \textbf{[Opción 5]}
\end{enumerate}

La captura generada se guarda automáticamente en el directorio desde el que se llama a la aplicación. Tras la captura, es recomendable almacenar dicho archivo, ya que si se vuelve a realizar una nueva, este archivo será sobrescribido.

%* Comprobar
\section{Conversión de la captura a formato \emph{PCAP}}
Para realizar la conversión a formato \emph{PCAP}, simplemente hay que llamar a la herramienta \emph{\$ \textbf{capture-json2pcap <ruta archivo JSON>}}, indicando como parámetro la ruta del archivo \emph{JSON}. El archivo convertido, se guarda en el directorio donde se encuentra la captura original, manteniendo el mismo nombre.