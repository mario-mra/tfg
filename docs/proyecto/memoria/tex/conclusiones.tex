\chapter{Conclusiones}
\label{ch:conclusiones}

%* Listo
En este capítulo se plasman las principales conclusiones del presente trabajo, y a su vez, se proponen posibles mejoras y trabajos futuros de interés.

%* Listo
\section{Conclusiones}
A continuación se detallan las principales conclusiones obtenidas durante la realización de este proyecto:

\begin{itemize}
    %* Listo
    \item Se ha procedido, en un primer lugar, a un análisis de mercado, confirmando la escasez de analizadores USB \emph{hardware} de bajo coste. Reafirmando de esa forma la finalidad del proyecto.
    
    %* Listo
    \item Se ha procedido a la selección y posterior conexión de los diversos componentes utilizados. Estos incluyen la placa de evaluación \emph{USB3300}, la placa de desarrollo \emph{iCEstick}, cableado de conexión, botonera y una base impresa en 3D.
    
    %* Listo
    \item Todo ello ha supuesto un precio de aproximadamente 45\$ (40\texteuro~al cambio), lo que supone una gran diferencia respecto a los analizadores comerciales, y una reducción de 43\$ respecto al proyecto de \emph{Ultra-Embedded} de similares características.
    
    %* Listo
    \item Se ha establecido una metodología de diseño y elaboración de módulos, que ha permitido una gran eficiencia en su elaboración, consiguiendo a su vez, una menor probabilidad de fallos y una alta reutilización.
    
    %* Listo
    \item El sistema soportar tramas USB 2.0 \emph{low speed}, \emph{full speed} y \emph{high speed}. Para esta última, debido a la escasa cantidad de memoria RAM disponible, puede haber perdida de datos si ocurre una gran transferencia USB en un breve periodo de tiempo.

    %* Listo
    \item Para el funcionamiento del sistema, se han diseñado y elaborado varios módulos, incluidos los establecidos en los objetivos, en el lenguaje descriptor de \emph{hardware} Verilog. \\
    Todos ellos han utilizado el 63\% de los bloques lógicos y el 100\% de los bloques de RAM disponibles en la FPGA usada. \\
    Principalmente hay que destacar los siguientes módulos.
    \begin{enumerate}
        \item Memoria FIFO.
        \item Registro de desplazamiento.
        \item Módulo de comunicación ULPI.
        \item Módulo de comunicación serie.
        \item Módulos generadores de relojes.
        \item Controlador maestro del sistema.
    \end{enumerate}

    %* Listo
    \item Se ha desarrollado una aplicación con las todas opciones necesarias para poder configurar la FPGA y recibir una captura, siendo esta almacenada en un archivo \emph{JSON} de fácil lectura.
    
    %* Listo
    \item Posteriormente, y para cumplir el objetivo propuesto, se ha creado una herramienta que convierte la trama capturada de formato \emph{JSON} a \emph{PCAP}.
    
    %* Listo
    \item Como observación final, hay que comentar varios aspectos a destacar en el uso de las FPGAs en general.
    \begin{itemize}
        \item Permiten una gran flexibilidad a la hora de realizar cualquier diseño de electrónica digital, y a diferencias de otros elementos como microcontroladores, permiten un alto nivel de paralelización de tareas, por lo que son de gran utilidad en ciertos diseños.
        
        \item Como contrapartida, las FPGAs introducen la necesidad de controlar las frecuencias máximas que pueden soportar, y una dificultad extra a la hora de realizar depuraciones.
    \end{itemize}
\end{itemize}

%* Listo
Con todo lo anteriormente expuesto, se puede afirmar que se han cumplido todos los objetivos establecidos al principio del trabajo, recogidos en el capítulo~\ref{ch:objetivos}.



%* Listo
\section{Trabajos futuros}
A continuación se enumeran varios trabajos futuros que se pueden realizar para mejorar el resultado del proyecto.

\begin{itemize}
    %* Listo
    \item Debido a la baja cantidad de memoria disponible en la FPGA usada, puede ser de gran interés añadir un módulo de memoria RAM externa, en la que almacenar de forma temporal los datos.
    
    %* Listo
    \item Implementación de un nuevo sistema de comunicación entre la FPGA y el PC, que prescindiendo del puerto serie actual, permita una mayor tasa de transferencia.
    
    %* Listo
    \item Diseño de una PCB que unifique los componentes utilizados en una sola placa, y se deshaga de todos aquellos innecesarios, como puede ser el emisor/receptor de infrarrojos de la placa iCEstick. Realizando esta tarea, a parte de obtener un producto más refinado, se puede disminuir aun más el precio final del sistema.
    
    %* Listo
    \item Creación de un disector para la utilidad de análisis \emph{Wireshark}, que sea capaz de mostrar más detalles de los datos capturados en el archivo \emph{PCAP}.
    
    %* Listo
    \item Introducción de herramientas de filtrado USB en la propia FPGA, que facilite el análisis. Por ejemplo, se podrían añadir filtros según el PID o la longitud de los datos a capturar.

    %* Listo
    \item Introducción de un sistema que permita saber en que instante temporal a ocurrido la captura de cada paquete de datos. Para ello, se puede añadir por ejemplo un circuito RTC (\emph{Real Time Clock}, o Reloj en Tiempo Real en español) que añada una marca temporal a cada paquete de datos obtenido.
\end{itemize}


% \chapter{Conclusiones}
% \label{ch:conclusiones-plantilla}

% Las conclusiones deben cerrar el documento, destacando los aspectos más importantes de la ejecución del TFG.  Debe analizar qué objetivos se han alcanzado y en qué grado, qué objetivos se han tenido que dejar fuera del proyecto y por qué, y qué líneas de trabajo futuro abre el TFG.  Para esta última parte es especialmente útil la lista \emph{Blocked} y la lista \emph{Backlog} de tu tablero Trello.

% En total las conclusiones no deberían superar dos o tres hojas.  Si ves que se alarga demasiado, traslada material al capítulo de discusión de resultados.

% Fíjate en que los objetivos abren el trabajo personal y las conclusiones lo cierran.  Procura mantener un orden que resalte esta relación.